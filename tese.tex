% Exemplo de dissertação do INF-UFG com texto em portugues formatado com LaTeX
\documentclass[dissertacao, abnt]{inf-ufg}
% Opções da classe inf-ufg (ao usar mais de uma, separe por vírgulas)
%   [tese]         -> Tese de doutorado.
%   [dissertacao]  -> Dissertação de mestrado (padrão).
%   [monografia]   -> Monografia de especialização.
%   [relatorio]    -> Relatório final de graduação.
%   [abnt]         -> Usa o estilo "abnt-alf" de citação bibliográfica.
%   [nocolorlinks] -> Os links de navegação no texto ficam na cor preta.
%                     Use esta opção para gerar o arquivo para impressão
%                     da versão final do seu texto!!!

%----------------------------------------------------- INICIO DO DOCUMENTO %

\usepackage[round,authoryear]{natbib}
\usepackage{arydshln}
\usepackage{booktabs}

\begin{document}

%------------------------------------------ AUTOR, TÍTULO E DATA DE DEFESA %
\autor{Fulano de Tal}
\autorR{Tal, Fulano de}

\titulo{Título do Trabalho}
% Predição de diâmetros de Clone do Eucalipto para o Cálculo de Volume Utilizando Redes Neurais de Múltiplas Camadas}
\subtitulo{Subtítulo do Trabalho}

\cidade{Goiânia}
\dia{15}
\mes{08}
\ano{2019} % Formato numérico: \dia{01}, \mes{01} e \ano{2009}

%-------------------------------------------------------------- ORIENTADOR %
\orientador{Tal, Fulano de}
\orientadorR{Tal, Fulano de}
% Use os comandos a seguir se for Orientadora e nao Orientador.
%\orientadora{\textless Nome da Orientadora\textgreater}
%\orientadoraR{\textless Nome Reverso da Orientadora\textgreater}

%\coorientador{\textless Nome do Co-orientador\textgreater}
%\coorientadorR{\textless Nome Reverso do Co-orienta dor\textgreater}
% Use os comandos a seguir se for Co-orientadora e nao Coorientador.
%\coorientadora{\textless Nome da Co-orientadora\textgreater}
%\coorientadoraR{\textless Nome Reverso da Co-orientadora\textgreater}

%-------------------------------------------------- INSTITUIÇÃO E PROGRAMA %
\universidade{Universidade Federal de Goiás}
\uni{UFG}
\unidade{Instituto de Informática}
\departamento{} % Unidades com mais de um departamento.

%\universidadeco{\textless Nome da Universidade do Co-orientador\textgreater}
%\unico{\textless Sigla da Universidade do Co-orientador\textgreater}
%\unidadeco{\textless Nome da Unidade Acadêmica do Co-orientador\textgreater}

\programa{Ciência da Computação}
\concentracao{Ciência da Computação}

%-------------------------------------------------- ELEMENTOS PRÉ-TEXTUAIS %
\capa    % Gera o modelo da capa externa do trabalho
%\publica % Gera a autorização para publicação em formato eletrônico

\cleardoublepage
% \includepdf[pages=-,templatesize={145mm}{210mm},noautoscale=true,offset=-100 -195]{teca.pdf}
%\includepdf[pages=1]{./tex/ata_digitalizada.pdf}
\rosto   % Primeira folha interna do trabalho

\input{./pre/pre_biblioteca}  %Ficha Catalográfica expedida pela secretaria do programa
\begin{aprovacao}
\banca{Dr. Fulano  de Tal}{Instituto de Ciências Agrárias\ -- UFMG}
\banca{Dr. Fulano de Tal}{Instituto de Informática\ -- UFG}
% Use o comando \profa se o membro da banca for do sexo feminino.
\end{aprovacao}
\input{./pre/pre_direitos}
\input{./pre/pre_dedicatoria}
\begin{agradecimentos}
<Texto com agradecimentos àquelas pessoas/entidades que, na opinião do autor,deram alguma contribuíção relevante para o desenvolvimento do trabalho.
\end{agradecimentos}



\input{./pre/pre_epigrafe}
\input{./pre/pre_resumo}
\keys{Palavra-chave 1, Palavra-chave 2, Palavra-chave 3, Palavra-chave 4}

\begin{abstract}{\textless Work title\textgreater}
Lorem ipsum dolor sit amet, consectetur adipiscing elit, sed do eiusmod tempor incididunt ut labore et dolore magna aliqua. Ut diam quam nulla porttitor massa id neque aliquam. Nec ultrices dui sapien eget mi proin sed libero. Magna fermentum iaculis eu non diam phasellus vestibulum. In hendrerit gravida rutrum quisque non tellus orci ac auctor. Dolor magna eget est lorem ipsum dolor sit. Egestas pretium aenean pharetra magna ac placerat vestibulum lectus mauris. Consequat semper viverra nam libero justo laoreet sit amet cursus. Maecenas pharetra convallis posuere morbi leo urna. Libero id faucibus nisl tincidunt. Viverra adipiscing at in tellus integer. Morbi non arcu risus quis varius quam quisque id. Et magnis dis parturient montes nascetur. Mi proin sed libero enim sed faucibus turpis in. Quis ipsum suspendisse ultrices gravida dictum fusce.
Pharetra et ultrices neque ornare aenean euismod elementum nisi. Donec enim diam vulputate ut pharetra sit amet aliquam. Aliquet bibendum enim facilisis gravida. Arcu cursus euismod quis viverra nibh. Quis vel eros donec ac odio tempor orci. Eget nulla facilisi etiam dignissim diam quis enim lobortis. Purus faucibus ornare suspendisse sed nisi lacus sed. Volutpat lacus laoreet non curabitur gravida. Sed augue lacus viverra vitae. Volutpat sed cras ornare arcu. Enim nec dui nunc mattis enim ut tellus. Tempus urna et pharetra pharetra massa. Dignissim convallis aenean et tortor at risus viverra adipiscing at. Risus at ultrices mi tempus imperdiet nulla malesuada pellentesque elit.
\end{abstract}


\tabelas[figtab]
%Opções:
%nada [] -> Gera apenas o sumário
%fig     -> Gera o sumário e a lista de figuras
%tab     -> Sumário e lista de tabelas
%alg     -> Sumário e lista de algoritmos
%cod     -> Sumário e lista de códigos de programas
%
% Pode-se usar qualquer combinação dessas opções.
% Por exemplo:
%  figtab       -> Sumário e listas de figuras e tabelas
%  figtabcod    -> Sumário e listas de figuras, tabelas e
%                  códigos de programas
%  figtabalg    -> Sumário e listas de figuras, tabelas e algoritmos
%  figtabalgcod -> Sumário e listas de figuras, tabelas, algoritmos e
%                  códigos de programas

%--------------------------------------------------------------- CAPÍTULOS %
\chapter{Introdução}
\label{cap:intro}

%\section{Problemática}
%Este é um exemplo de introdução com citação \Citep{Arm1979}.
%\section{Objetivos}
%\section{Resumo}

\section{Motivação}

Lorem Ipsum is simply dummy text of the printing and typesetting industry. Lorem Ipsum has been the industry's standard dummy text ever since the 1500s, when an unknown printer took a galley of type and scrambled it to make a type specimen book. It has survived not only five centuries, but also the leap into electronic typesetting, remaining essentially unchanged. It was popularised in the 1960s with the release of Letraset sheets containing Lorem Ipsum passages, and more recently with desktop publishing software like Aldus PageMaker including versions of Lorem Ipsum \Citep{soares2010}

Lorem Ipsum \Citet{soares2010} is simply dummy text of the printing and typesetting industry. Lorem Ipsum has been the industry's standard dummy text ever since the 1500s, when an unknown printer took a galley of type and scrambled it to make a type specimen book. It has survived not only five centuries, but also the leap into electronic typesetting, remaining essentially unchanged. It was popularised in the 1960s with the release of Letraset sheets containing Lorem Ipsum passages, and more recently with desktop publishing software like Aldus PageMaker including versions of Lorem Ipsum
Lorem Ipsum is simply dummy text of the printing and typesetting industry. Lorem Ipsum has been the industry's standard dummy text ever since the 1500s, when an unknown printer took a galley of type and scrambled it to make a type specimen book. It has survived not only five centuries, but also the leap into electronic typesetting, remaining essentially unchanged. It was popularised in the 1960s with the release of Letraset sheets containing Lorem Ipsum passages, and more recently with desktop publishing software like Aldus PageMaker including versions of Lorem Ipsum

Lorem Ipsum is simply dummy text of the printing and typesetting industry. Lorem Ipsum has been the industry's standard dummy text ever since the 1500s, when an unknown printer took a galley of type and scrambled it to make a type specimen book. It has survived not only five centuries, but also the leap into electronic typesetting, remaining essentially unchanged. It was popularised in the 1960s with the release of Letraset sheets containing Lorem Ipsum passages, and more recently with desktop publishing software like Aldus PageMaker including versions of Lorem Ipsum

Lorem Ipsum is simply dummy text of the printing and typesetting industry. Lorem Ipsum has been the industry's standard dummy text ever since the 1500s, when an unknown printer took a galley of type and scrambled it to make a type specimen book. It has survived not only five centuries, but also the leap into electronic typesetting, remaining essentially unchanged. It was popularised in the 1960s with the release of Letraset sheets containing Lorem Ipsum passages, and more recently with desktop publishing software like Aldus PageMaker including versions of Lorem Ipsum

\section{Objetivos}

Lorem Ipsum is simply dummy text of the printing and typesetting industry. Lorem Ipsum has been the industry's standard dummy text ever since the 1500s, when an unknown printer took a galley of type and scrambled it to make a type specimen book. It has survived not only five centuries, but also the leap into electronic typesetting, remaining essentially unchanged. It was popularised in the 1960s with the release of Letraset sheets containing Lorem Ipsum passages, and more recently with desktop publishing software like Aldus PageMaker including versions of Lorem Ipsum

\subsection{Objetivos específicos}

Lorem Ipsum is simply dummy text of the printing and typesetting industry. Lorem Ipsum has been the industry's standard dummy text ever since the 1500s, when an unknown printer took a galley of type and scrambled it to make a type specimen book. It has survived not only five centuries, but also the leap into electronic typesetting, remaining essentially unchanged. It was popularised in the 1960s with the release of Letraset sheets containing Lorem Ipsum passages, and more recently with desktop publishing software like Aldus PageMaker including versions of Lorem Ipsum

\begin{itemize}
\item Lorem Ipsum is simply dummy text of the printing and typesetting industry.
\item Lorem Ipsum is simply dummy text of the printing and typesetting industry.
\item Lorem Ipsum is simply dummy text of the printing and typesetting industry.
\item Lorem Ipsum is simply dummy text of the printing and typesetting industry.
\item Lorem Ipsum is simply dummy text of the printing and typesetting industry.
\item Lorem Ipsum is simply dummy text of the printing and typesetting industry.
\end{itemize}

\section{Organização do trabalho}

\chapter{Referencial Teórico}
\label{cap:revbib}

Lorem Ipsum is simply dummy text of the printing and typesetting industry. Lorem Ipsum has been the industry's standard dummy text ever since the 1500s, when an unknown printer took a galley of type and scrambled it to make a type specimen book. It has survived not only five centuries, but also the leap into electronic typesetting, remaining essentially unchanged. It was popularised in the 1960s with the release of Letraset sheets containing Lorem Ipsum passages, and more recently with desktop publishing software like Aldus PageMaker including versions of Lorem Ipsum

Lorem Ipsum is simply dummy text of the printing and typesetting industry. Lorem Ipsum has been the industry's standard dummy text ever since the 1500s, when an unknown printer took a galley of type and scrambled it to make a type specimen book. It has survived not only five centuries, but also the leap into electronic typesetting, remaining essentially unchanged. It was popularised in the 1960s with the release of Letraset sheets containing Lorem Ipsum passages, and more recently with desktop publishing software like Aldus PageMaker including versions of Lorem Ipsum

Lorem Ipsum is simply dummy text of the printing and typesetting industry. Lorem Ipsum has been the industry's standard dummy text ever since the 1500s, when an unknown printer took a galley of type and scrambled it to make a type specimen book. It has survived not only five centuries, but also the leap into electronic typesetting, remaining essentially unchanged. It was popularised in the 1960s with the release of Letraset sheets containing Lorem Ipsum passages, and more recently with desktop publishing software like Aldus PageMaker including versions of Lorem Ipsum


\section{Section}

Lorem Ipsum is simply dummy text of the printing and typesetting industry. Lorem Ipsum has been the industry's standard dummy text ever since the 1500s, when an unknown printer took a galley of type and scrambled it to make a type specimen book. It has survived not only five centuries, but also the leap into electronic typesetting, remaining essentially unchanged. It was popularised in the 1960s with the release of Letraset sheets containing Lorem Ipsum passages, and more recently with desktop publishing software like Aldus PageMaker including versions of Lorem Ipsum


\begin{figure}[ht]
\centering
\includegraphics[width=\textwidth]{fig/exemploFig1.eps}
\caption{Lorem Ipsum \Citep{anual2017industria}.}
\label{fig:mapa_brasil}
\end{figure}

Lorem Ipsum is simply dummy text of the printing and typesetting industry. Lorem Ipsum has been the industry's standard dummy text ever since the 1500s, when an unknown printer took a galley of type and scrambled it to make a type specimen book. It has survived not only five centuries, but also the leap into electronic typesetting, remaining essentially unchanged. It was popularised in the 1960s with the release of Letraset sheets containing Lorem Ipsum passages, and more recently with desktop publishing software like Aldus PageMaker including versions of Lorem Ipsum
\chapter{Capítulo III}
\label{cap:meto}

Lorem Ipsum is simply dummy text of the printing and typesetting industry. Lorem Ipsum has been the industry's standard dummy text ever since the 1500s, when an unknown printer took a galley of type and scrambled it to make a type specimen book. It has survived not only five centuries, but also the leap into electronic typesetting, remaining essentially unchanged. It was popularised in the 1960s with the release of Letraset sheets containing Lorem Ipsum passages, and more recently with desktop publishing software like Aldus PageMaker including versions of Lorem Ipsum

Lorem Ipsum is simply dummy text of the printing and typesetting industry. Lorem Ipsum has been the industry's standard dummy text ever since the 1500s, when an unknown printer took a galley of type and scrambled it to make a type specimen book. It has survived not only five centuries, but also the leap into electronic typesetting, remaining essentially unchanged. It was popularised in the 1960s with the release of Letraset sheets containing Lorem Ipsum passages, and more recently with desktop publishing software like Aldus PageMaker including versions of Lorem Ipsum

\begin{equation}
h_{t} = \phi(b + Wh_{t-1} + Ux_{t})
\label{eq:expand_vanilla_rnn}
\end{equation}

Lorem Ipsum is simply dummy text of the printing and typesetting industry. Lorem Ipsum has been the industry's standard dummy text ever since the 1500s, when an unknown printer took a galley of type and scrambled it to make a type specimen book. It has survived not only five centuries, but also the leap into electronic typesetting, remaining essentially unchanged. It was popularised in the 1960s with the release of Letraset sheets containing Lorem Ipsum passages, and more recently with desktop publishing software like Aldus PageMaker including versions of Lorem Ipsum

\begin{equation}
    o_{t} = c + Vh_{t}
\label{eq:output_vanilla_rnn}
\end{equation}

Lorem Ipsum is simply dummy text of the printing and typesetting industry. Lorem Ipsum has been the industry's standard dummy text ever since the 1500s, when an unknown printer took a galley of type and scrambled it to make a type specimen book. It has survived not only five centuries, but also the leap into electronic typesetting, remaining essentially unchanged. It was popularised in the 1960s with the release of Letraset sheets containing Lorem Ipsum passages, and more recently with desktop publishing software like Aldus PageMaker including versions of Lorem Ipsum

\begin{equation}
E(y_{t},\hat{y_{t}}) & = \sum_{t} L_{t}(y_{t}, \hat{y}_{t})
\label{eq:passo2}
\end{equation}

\begin{equation}
= - \sum_{t} y_{t}\log \hat{y}_{t} 
    \label{eq:cost_rnn}
\end{equation}
Lorem Ipsum is simply dummy text of the printing and typesetting industry. Lorem Ipsum has been the industry's standard dummy text ever since the 1500s, when an unknown printer took a galley of type and scrambled it to make a type specimen book. It has survived not only five centuries, but also the leap into electronic typesetting, remaining essentially unchanged. It was popularised in the 1960s with the release of Letraset sheets containing Lorem Ipsum passages, and more recently with desktop publishing software like Aldus PageMaker including versions of Lorem Ipsum

\begin{equation}
\frac{\partial E}{\partial W} = \sum^{T}_{t=1}\frac{\partial E_{t}}{\partial W}
    \label{eq:totalerror}
\end{equation}

Lorem Ipsum is simply dummy text of the printing and typesetting industry. Lorem Ipsum has been the industry's standard dummy text ever since the 1500s, when an unknown printer took a galley of type and scrambled it to make a type specimen book. It has survived not only five centuries, but also the leap into electronic typesetting, remaining essentially unchanged. It was popularised in the 1960s with the release of Letraset sheets containing Lorem Ipsum passages, and more recently with desktop publishing software like Aldus PageMaker including versions of Lorem Ipsum

\begin{equation}
s_{t} = \phi(Ws_{t-1} + Ux_{t})
\label{eq:mod_rec}
\end{equation}

\begin{equation}
    \hat{y}_{t} = \textup{softmax}(Vs_{t})
\label{eq:mod_softmax}
\end{equation}

\begin{equation}
    \frac{\partial E_{t}}{\partial V} = \frac{\partial E_{t}}{\partial \hat{y}_{t}}\frac{\partial \hat{y}_{t}}{\partial V}
\label{eq:mod_softmax}
\end{equation}

\begin{equation}
\frac{\partial E_{t}}{\partial V} = \frac{\partial E_{3}}{\partial \hat{y}_{t}}\frac{\partial \hat{y}_{t}}{\partial z_{t}}\frac{\partial z_{t}}{\partial V}
\label{eq:mod_softmax}
\end{equation}

\begin{equation}
    \frac{\partial E_{t}}{\partial V} = (\hat{y}_{t} - y_{t}) \otimes s_{t}
\label{eq:partial_v}
\end{equation}

Lorem Ipsum is simply dummy text of the printing and typesetting industry. Lorem Ipsum has been the industry's standard dummy text ever since the 1500s.

\begin{equation}
\frac{\partial E_{t}}{\partial W} = \sum^{t}_{k=0}\frac{\partial E_{t}}{\partial \hat{y}_{t}}\frac{\partial \hat{y}_{t}}{\partial s_{t}}\frac{\partial s_{t}}{\partial s_{k}}\frac{\partial s_{k}}{\partial W}
\label{eq:rule_chain}
\end{equation}

Lorem Ipsum is simply dummy text of the printing and typesetting industry. Lorem Ipsum has been the industry's standard dummy text ever since the 1500s, when an unknown printer took a galley of type and scrambled it to make a type specimen book

\begin{equation}
\delta^{t}_{k} = \frac{\partial E_{t} }{\partial z_{k}} = \frac{\partial E_{t} }{\partial s_{k}}\frac{\partial s_{t} }{\partial s_{k}}\frac{\partial s_{k} }{\partial z_{k}}
\label{eq:deltarnn}
\end{equation}


\input{./tex/cap_4}
\chapter{Metodologia}
\label{cap:matandmeth}

Lorem Ipsum is simply dummy text of the printing and typesetting industry. Lorem Ipsum has been the industry's standard dummy text ever since the 1500s, when an unknown printer took a galley of type and scrambled it to make a type specimen book. It has survived not only five centuries, but also the leap into electronic typesetting, remaining essentially unchanged. It was popularised in the 1960s with the release of Letraset sheets containing Lorem Ipsum passages, and more recently with desktop publishing software like Aldus PageMaker including versions of Lorem Ipsum.

\section{Materiais}

Lorem Ipsum is simply dummy text of the printing and typesetting industry. Lorem Ipsum has been the industry's standard dummy text ever since the 1500s, when an unknown printer took a galley of type and scrambled it to make a type specimen book. It has survived not only five centuries, but also the leap into electronic typesetting, remaining essentially unchanged. It was popularised in the 1960s with the release of Letraset sheets containing Lorem Ipsum passages, and more recently with desktop publishing software like Aldus PageMaker including versions of Lorem Ipsum.


\section{Métodos}

Lorem Ipsum is simply dummy text of the printing and typesetting industry. Lorem Ipsum has been the industry's standard dummy text ever since the 1500s, when an unknown printer took a galley of type and scrambled it to make a type specimen book. It has survived not only five centuries, but also the leap into electronic typesetting, remaining essentially unchanged. It was popularised in the 1960s with the release of Letraset sheets containing Lorem Ipsum passages, and more recently with desktop publishing software like Aldus PageMaker including versions of Lorem Ipsum.

\subsection{Lorem Ipsum}

Lorem Ipsum is simply dummy text of the printing and typesetting industry. Lorem Ipsum has been the industry's standard dummy text ever since the 1500s, when an unknown printer took a galley of type and scrambled it to make a type specimen book. It has survived not only five centuries, but also the leap into electronic typesetting, remaining essentially unchanged. It was popularised in the 1960s with the release of Letraset sheets containing Lorem Ipsum passages, and more recently with desktop publishing software like Aldus PageMaker including versions of Lorem Ipsum.


\begin{table}[h!tp]

\centering
\caption{Recursos utilizados.}
% \resizebox{0.7\textwidth}{!}{
\begin{tabular}{ll}
\toprule
\textbf{Recursos}           & \textbf{Descrição}                                                                                       \\ \midrule
Processador                 & Intel i7 de 5º geração                                                                                   \\
Memória Ram                 & 8 GB                                                                                                     \\
Sitema Operacional          & Debian 9 Stretch                                                                                         \\
Ambiente de Desenvolvimento & \begin{tabular}[c]{@{}l@{}}Python 3.5, Spyder IDE, Jupyter Notebook,\\ Sklearn e Tensorflow\end{tabular} \\ \bottomrule
\end{tabular}
\label{tbl:recursos}
% }
\end{table}

Lorem Ipsum is simply dummy text of the printing and typesetting industry. Lorem Ipsum has been the industry's standard dummy text ever since the 1500s, when an unknown printer took a galley of type and scrambled it to make a type specimen book. It has survived not only five centuries, but also the leap into electronic typesetting, remaining essentially unchanged. It was popularised in the 1960s with the release of Letraset sheets containing Lorem Ipsum passages, and more recently with desktop publishing software like Aldus PageMaker including versions of Lorem Ipsum.

Lorem Ipsum is simply dummy text of the printing and typesetting industry. Lorem Ipsum has been the industry's standard dummy text ever since the 1500s, when an unknown printer took a galley of type and scrambled it to make a type specimen book. It has survived not only five centuries, but also the leap into electronic typesetting, remaining essentially unchanged. It was popularised in the 1960s with the release of Letraset sheets containing Lorem Ipsum passages, and more recently with desktop publishing software like Aldus PageMaker including versions of Lorem Ipsum.


\input{./tex/cap_6}
\input{./tex/cap_7}
% \chapter{Considerações Finais}
\label{cap:consideracoes_finais}

Os resultados alcançados até o presente momento mostraram-se promissores para dar continuidade a investigação, visto que a primeira fase se melhorara os resultados existentes já fora realizada.

Os passos seguintes serão essenciais para apresentar outras metodologias que visem aperfeiçoar as escolhas de parâmetros para a MLP, buscando principalmente minimizar os erros durante a obtenção dos dados para a avaliação pela MLP, diminuindo as complexidades de obter tais dados em campo como, por exemplo, a altura total da árvore, além de contribuir para um inventário florestal mais preciso.

Após a conclusão deste trabalho, o estudo de avaliação de ajustes de parâmetros para redes neurais pretende contribuir para minimizar os erros na escolha de parâmetros da MLP, como também, buscar técnicas que visem generalizar o modelo treinado para demais sítios.

%\chapter{Introdução}
\label{cap:intro}

Este é um exemplo de introdução com citação \cite{Arm1979}.


%\input{./tex/cap_II}
%\input{./tex/cap_III}

%------------------------------------------------------------ BIBLIOGRAFIA %
\cleardoublepage
%\nocite{*} %%% Retire esta linha para gerar a bibliografia com apenas as
           %%% referências usadas no seu texto!
\arial
\bibliography{./bib/tese} %%% Nomes dos seus arquivos .bib
\label{ref-bib}

%--------------------------------------------------------------- APÊNDICES %
%\apendices

%\input{./pos/apend_I}
%\input{./pos/apend_II}

\end{document}

%------------------------------------------------------------------------- %
%        F I M   D O  A R Q U I V O :  m o d e l o - t e s e . t e x       %
%------------------------------------------------------------------------- %
